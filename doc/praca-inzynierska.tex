\documentclass[polish,12pt]{aghthesis}
% Szablon przystosowany jest do druku dwustronnego. 

\usepackage[utf8]{inputenc}
\usepackage{url}

\author{Gabriela Kowalik, Mateusz Olczyk, Rafał Pietruszka}
\titlePL{Aplikacja demonstrująca możliwości eskalacji uprawnień i wydobywania danych chronionych w systemie Android}
\titleEN{Application demonstrating possibilities of escalating permissions and extracting protected data in the Android
    system}
\fieldofstudy{Informatyka}
\supervisor{mgr \ inż.\ Łukasz Faber}
\date{\the\year}

\begin{document}

\maketitle

\section{\SectionTitleProjectVision}
\label{sec:cel-wizja}
% Charakterystyka problemu, motywacja projektu (w tym przegląd istniejących rozwiązań prowadząca do uzasadnienia celu
% prac), wizja produktu i analiza zagrożeń.
\subsection{Dziedzina problemu}
Według statystyk firmy \emph{StatCounter} \cite{article_mobile_os_market_stats} z września 2018 roku na rynku mobilnych
systemów operacyjnych dominuje system \emph{Android} z wynikiem 76,61\%.
Na drugim miejscu plasuje się system \emph{iOS} z wynikiem 20,66\%.
Oznacza to, że Android ma znaczącą przewagę na rynku i z tego powodu jest popularnym celem hakerów zamierzających
uruchamiać na urządzeniach nieświadomych użytkowników złośliwe oprogramowanie.

W tworzeniu Androida ma swój udział wiele grup.
Jest to firma Google odpowiadająca za szkielet systemu operacyjnego i zarządzaniem marką Android, producenci układów
scalonych, producenci sprzętu oraz operatorzy sieci komórkowych.
Złożoność tego ekosystemu utrudnia zapewnienie bezpieczeństwa, ponieważ różne grupy mają różne cele które bywają dla
siebie konkurencyjne.

Efektem takiej złożoności jest jedna z najważniejszych komplikacji Androida, czyli problemy z aktualizacją.
Przykładowo rządzenia typu Nexus mogą otrzymać łatę dotyczącą pewnego problemu w systemie niemal natychmiast, gdy Google
naprawi ten problem.
Niestety urządzeń z oprogramowaniem modyfikowanym przez producenta to nie dotyczy.
Taka sytuacja wymaga od producenta nowej kompilacji oprogramowania z uwzględnieniem łaty od Google.
Podobnie jest w przypadku oprogramowania zmodyfikowanego przez operatora sieci komórkowej.
Przez taką komplikację użytkownik otrzymuje aktualizację systemu operacyjnego dużo później.
Ponadto w systemi Android praktycznie nie istnieją \textit{backporty}, czyli poprawki aktualnej wersji oprogramowania
nanoszone na starszą wersję systemu.

Jeszcze jednym problemem związanym z bezpieczeństwem systemu Android jest upublicznianie informacji na temat wykrytych
podatności.
W ekosystemie Androida niestety niezwykle rzadko producenci publicznie ujawniają takie informacje.

Dlatego obraliśmy sobie za cel zbadanie bezpieczeństwa tego najpopularniejszego mobilnego systemu operacyjnego.
Pod uwagę wzięliśmy wersje od 4.0 (ICS) do najnowszej dostępnej. W tej pracy zmierzymy się z próbą wykazania różnych
podatności systemu.

\subsection{Minimal viable product}
Naszym celem było stworzenie aplikacji na Androida składającej się z kilku niezależnych modułów.
Moduły mają demonstrować kilka technik dostępu do danych bez uprawnień roota oraz kilka technik dostępu do danych z
przydzielonymi uprawnieniami roota.

\subsection{Analiza ryzyka}
\begin{tabular}{|p{9cm}|c|c|}
\hline
    \textbf{Zagrożenie} &
    \textbf{Prawdopodobieństwo} &
    \textbf{Skutki} \\ 
\hline
    Wydawane aktualizacje bezpieczeństwa mogą blokować dostępne wektory ataku, szczególnie w przypadku podatności
    krytycznych. &
    6 &
    10 \\
\hline
    Dostarczane przez poszczególnych producentów platformy mogą znacząco odbiegać od oficjalnych specyfikacji systemu
    Android np. konfiguracja polityki bezpieczeństwa, zmodyfikowany firmware / bootloader / aplikacje. &
    5 &
    8 \\
\hline
    Dostępność urządzeń oraz poszczególnych wersji systemu Android. &
    4 &
    5 \\
\hline
    Operacje wykradania danych są zazwyczaj złożone i oparte na nietypowych "trikach". Z tego powodu nie są przewidziane
    ani udokumentowane przez producenta systemu i może brakować odpowiednich źródeł wiedzy umożliwiających ich
    odtworzenie. &
    4 &
    6 \\
    \hline
\end{tabular}


\section{\SectionTitleScope}
\label{sec:zakres-funkcjonalnosci}
% Kontekst użytkowania produktu (aktorzy, współpracujące systemy) oraz specyfikacja wymagań funkcjonalnych i
% niefunkcjonalnych.
\subsection{Wymagania funkcjonalne}
\begin{tabular}{|p{12cm}|c|}
\hline
    \textbf{Wymagania funkcjonalne} &
    \textbf{Priorytet} \\ 
\hline
    Demonstracja eskalacji uprawnień roota &
    10 \\
\hline
    Uzyskiwanie dostępu do danych chronionych &
    10 \\
\hline
    Funkcjonalność składowania uzyskanych danych na zdalnych serwerach &
    6 \\
\hline
    Implementacja rozwiązania zapewniającego zdalny dostęp do atakowanego urządzenia &
    4 \\
\hline
    Demonstracja socjotechnik wykorzystujących błędy użytkownika &
    2 \\
\hline
\end{tabular}

\subsection{Wymagania niefunkcjonalne}
\begin{tabular}{|p{12cm}|}
\hline
    \textbf{Wymagania niefunkcjonalne} \\ 
\hline
    Ograniczenie zakresu testów do dopuszczalnych przypadków prawnych. \\
\hline
    Wykorzystanie możliwe najnowszych wersji systemu Android. \\
\hline
    Zastosowanie technologii Java / Kotlin + Android SDK oraz Android NDK + C/C++ dla zabudowy natywnego kodu dla
    platformy docelowej. \\
\hline
\end{tabular}


\section{\SectionTitleRealizationAspects}
\label{sec:wybrane-aspekty-realizacji}
% Przyjęte założenia, struktura i zasada działania systemu, wykorzystane rozwiązania technologiczne wraz z uzasadnieniem
% ich wyboru, istotne mechanizmy i zastosowane algorytmy.
\subsection{Wykorzystanie zewnętrznych systemów i aplikacji}
W swojej pracy korzystaliśmy z następujących systemów i aplikacji:
\begin{itemize}
    \item Android Debug Bridge
    \item Android API
\end{itemize}


\section{\SectionTitleWorkOrganization}
\label{sec:organizacja-pracy}
% Struktura zespołu (role poszczególnych osób), krótki opis i uzasadnienie przyjętej metodyki i/lub kolejności prac,
% planowane i zrealizowane etapy prac ze wskazaniem udziału poszczególnych członków zespołu, wykorzystane praktyki i
% narzędzia w zarządzaniu projektem.
\input{organizacja-pracy.tex}

\section{\SectionTitleResults}
\label{sec:wyniki-projektu}
% Wskazanie wyników projektu (co konkretnie udało się uzyskać: oprogramowanie, dokumentacja, raporty z testów/wdrożenia,
% itd.), prezentacja wyników i ocena ich użyteczności (jak zostało to zweryfikowane - np. wnioski klienta/użytkownika,
% zrealizowane testy wydajnościowe, itd.), istniejące ograniczenia i propozycje dalszych prac.
\input{wyniki-projektu.tex}

\nocite{article_mobile_os_market_stats,
        book_android_podrecznik_hackera}
\bibliography{bibliografia}

\end{document}
