\documentclass[polish,12pt]{aghthesis}
% Szablon przystosowany jest do druku dwustronnego. 

\usepackage[utf8]{inputenc}
\usepackage{url}

\author{Gabriela Kowalik, Mateusz Olczyk, Rafał Pietruszka}

\titlePL{Aplikacja demonstrująca możliwości eskalacji uprawnień i wydobywania danych chronionych w systemie Android}
\titleEN{Application demonstrating possibilities of escalating permissions and extracting protected data in the Android
    system}

\fieldofstudy{Informatyka}

\supervisor{dr hab.\ inż.\ Krzysztof Iksiński, prof.\ nadzw.\ AGH}

\date{\the\year}

\begin{document}

\maketitle

\section{\SectionTitleProjectVision}
\label{sec:cel-wizja}
\emph{Charakterystyka problemu, motywacja projektu (w tym przegląd
    istniejących rozwiązań prowadząca do uzasadnienia celu prac),
    wizja produktu i analiza zagrożeń.}  % niniejsza linijka to tylko komentarz, który należy usunąć
\subsection{Dziedzina problemu}
Bezpieczeństwo najpopularniejszego systemu mobilnego, Androida.
Pod uwagi wzięte wersje od 4.0 (ICS) do najnowszego dostępnego.
\subsection{Minimal viable product}
Aplikacja (aplikacje) demonstrujące kilka technik dostępu do danych bez uprawnień roota oraz kilka technik dostępu do danych z przydzielonymi uprawnieniami roota.
\subsection{Analiza ryzyka}
\begin{tabular}{|c|c|c|}
    \hline
    \textbf{Zagrożenie} & \textbf{Prawdopodobieństwo} & \textbf{Skutki} \\ 
    \hline
    Wydawane aktualizacje bezpieczeństwa mogą blokować\\
    dostępne wektory ataku, szczególnie w przypadku\\
    podatności krytycznych & 6 & 10 \\
    \hline
    Dostarczane przez poszczególnych producentów\\
    platformy mogą znacząco	odbiegać od oficjalnych\\
    specyfikacji systemu Android np. konfiguracja\\
    polityki bezpieczeństwa, zmodyfikowany firmware\\
    / bootloader / aplikacje. & 5 & 8 \\
    \hline
    Dostępność urządzeń oraz poszczególnych wersji\\
    systemu Android & 4 & 5 \\
    \hline
    Operacje wykradania danych są zazwyczaj złożone i\\
    oparte na nietypowych "trikach". Z tego powodu nie\\
    są przewidziane ani udokumentowane przez producenta\\
    systemu i może brakować odpowiednich źródeł wiedzy\\
    umożliwiających ich odtworzenie. & 4 & 6 \\
    \hline
\end{tabular}

\section{\SectionTitleScope}
\label{sec:zakres-funkcjonalnosci}
\emph{Kontekst użytkowania produktu (aktorzy, współpracujące systemy)
  oraz specyfikacja wymagań funkcjonalnych i niefunkcjonalnych.}  % niniejsza linijka to tylko komentarz, który należy usunąć
\subsection{Wymagania funkcjonalne}
\begin{tabular}{|c|c|}
    \hline
    \textbf{Wymagania funkcjonalne} & \textbf{Priorytet} \\ 
    \hline
    Demonstracja eskalacji uprawnień roota & 10 \\
    \hline
    Uzyskiwanie dostępu do danych chronionych &	10 \\
    \hline
    Funkcjonalność składowania uzyskanych danych na zdalnych serwerach & 6 \\
    \hline
    Implementacja rozwiązania zapewniającego zdalny dostęp do atakowanego urządzenia & 4 \\
    \hline
    Demonstracja socjotechnik wykorzystujących błędy użytkownika & 2 \\
    \hline
\end{tabular}
\subsection{Wymagania niefunkcjonalne}
\begin{tabular}{|c|}
    \hline
    \textbf{Wymagania niefunkcjonalne} \\ 
    \hline
    Ograniczenie zakresu testów do dopuszczalnych przypadków prawnych. \\
    \hline
    Wykorzystanie możliwe najnowszych wersji systemu Android. \\
    \hline
    Zastosowanie technologii Java / Kotlin + Android SDK oraz Android NDK + C/C++\\dla zabudowy natywnego kodu dla
    platformy docelowej. \\
    \hline
\end{tabular} 


\section{\SectionTitleRealizationAspects}
\label{sec:wybrane-aspekty-realizacji}
\emph{Przyjęte założenia, struktura i zasada działania systemu,
  wykorzystane rozwiązania technologiczne wraz z uzasadnieniem
  ich wyboru, istotne mechanizmy i zastosowane algorytmy.} % niniejsza linijka to tylko komentarz, który należy usunąć
\subsection{Wykorzystanie zewnętrznych systemów i aplikacji}
Android Debug Bridge\\
Android API

\section{\SectionTitleWorkOrganization}
\label{sec:organizacja-pracy}
\emph{Struktura zespołu (role poszczególnych osób), krótki opis i
  uzasadnienie przyjętej metodyki i/lub kolejności prac, planowane i
  zrealizowane etapy prac ze wskazaniem udziału poszczególnych
  członków zespołu, wykorzystane praktyki i narzędzia w zarządzaniu
  projektem.}  % niniejsza linijka to tylko komentarz, który należy usunąć

\section{\SectionTitleResults}
\label{sec:wyniki-projektu}
\emph{Wskazanie wyników projektu (co konkretnie udało się uzyskać:
  oprogramowanie, dokumentacja, raporty z testów/wdrożenia, itd.), prezentacja wyników
  i ocena ich użyteczności (jak zostało to zweryfikowane --- np.\ wnioski
  klienta/użytkownika, zrealizowane testy wydajnościowe, itd.),
  istniejące ograniczenia i propozycje dalszych prac.}  % niniejsza linijka to tylko komentarz, który należy usunąć

% o ile to możliwe proszę uzywać odwołań \cite w konkretnych miejscach a nie \nocite

\nocite{artykul2011,ksiazka2011,narzedzie2011,projekt2011}

\bibliography{bibliografia}

\end{document}
