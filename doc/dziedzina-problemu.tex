\subsection{Dziedzina problemu}
Według statystyk firmy \emph{StatCounter} \cite{article_mobile_os_market_stats} z września 2018 roku na rynku mobilnych
systemów operacyjnych dominuje system \emph{Android} z wynikiem 76,61\%.
Na drugim miejscu plasuje się system \emph{iOS} z wynikiem 20,66\%.
Oznacza to, że Android ma znaczącą przewagę na rynku i z tego powodu jest popularnym celem hakerów zamierzających
uruchamiać na urządzeniach nieświadomych użytkowników złośliwe oprogramowanie.

W tworzeniu Androida ma swój udział wiele grup.
Jest to firma Google odpowiadająca za szkielet systemu operacyjnego i zarządzaniem marką Android, producenci układów
scalonych, producenci sprzętu oraz operatorzy sieci komórkowych.
Złożoność tego ekosystemu utrudnia zapewnienie bezpieczeństwa, ponieważ różne grupy mają różne cele które bywają dla
siebie konkurencyjne.

Efektem takiej złożoności jest jedna z najważniejszych komplikacji Androida, czyli problemy z aktualizacją.
Przykładowo rządzenia typu Nexus mogą otrzymać łatę dotyczącą pewnego problemu w systemie niemal natychmiast, gdy Google
naprawi ten problem.
Niestety urządzeń z oprogramowaniem modyfikowanym przez producenta to nie dotyczy.
Taka sytuacja wymaga od producenta nowej kompilacji oprogramowania z uwzględnieniem łaty od Google.
Podobnie jest w przypadku oprogramowania zmodyfikowanego przez operatora sieci komórkowej.
Przez taką komplikację użytkownik otrzymuje aktualizację systemu operacyjnego dużo później.
Ponadto w systemi Android praktycznie nie istnieją \textit{backporty}, czyli poprawki aktualnej wersji oprogramowania
nanoszone na starszą wersję systemu.

Jeszcze jednym problemem związanym z bezpieczeństwem systemu Android jest upublicznianie informacji na temat wykrytych
podatności.
W ekosystemie Androida niestety niezwykle rzadko producenci publicznie ujawniają takie informacje.

Dlatego obraliśmy sobie za cel zbadanie bezpieczeństwa tego najpopularniejszego mobilnego systemu operacyjnego.
Pod uwagę wzięliśmy wersje od 4.0 (ICS) do najnowszej dostępnej. W tej pracy zmierzymy się z próbą wykazania różnych
podatności systemu.
